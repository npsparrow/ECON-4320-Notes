\documentclass[letterpaper, 12pt, titlepage]{article}
\usepackage{hyperref}
\usepackage{graphicx}

\title{ECON 4320 Comprehensive Notes}
\author{Nikhil Panickssery}
\date{\today}

% ========================
%    FONTS AND ENCODING
% ========================

\usepackage{fourier}
\usepackage{opensans}
\usepackage[T1]{fontenc}

\usepackage[DIV=14,BCOR=2mm,headinclude=true,footinclude=false]{typearea}
\makeatletter
\if@twoside % commands below work only for twoside option of \documentclass
    \newlength{\textblockoffset}
    \setlength{\textblockoffset}{12mm}
    \addtolength{\hoffset}{\textblockoffset}
    \addtolength{\evensidemargin}{-2.0\textblockoffset}
\fi
\makeatother

\usepackage{titlesec} % www.ctan.org/tex-archive/macros/latex/contrib/titlesec/titlesec.pdf
\titleformat{name=\section}[hang]{
  \usefont{T1}{opensans-TLF}{b}{n}\selectfont}
  {} % label
  {0em} % horizontal separation between label and title body
  {\hspace{-0.4pt}\Large \thesection\hspace{0.6em}} % code preceding the title
\titleformat{name=\section,numberless}[hang]{
  \usefont{T1}{opensans-TLF}{b}{n}\selectfont}
  {} % label
  {0em} % horizontal separation between label and title body
  {\hspace{-0.4pt}\Large } % code preceding the title
\titleformat{\subsection}[hang]{
  \usefont{T1}{opensans-TLF}{b}{n}\selectfont}
  {} % label
  {0em} % horizontal separation between label and title body
  {\hspace{-0.4pt}\large \thesubsection\hspace{0.6em}} % code preceding the title
\titleformat{name=\subsection,numberless}[hang]{
  \usefont{T1}{opensans-TLF}{b}{n}\selectfont}
  {} % label
  {0em} % horizontal separation between label and title body
  {\hspace{-0.4pt}\large } % code preceding the title
\titleformat{\subsubsection}[hang]{
  \usefont{T1}{opensans-TLF}{b}{n}\selectfont}
  {} % label
  {0em} % horizontal separation between label and title body
  {\hspace{-0.4pt}\large \thesubsubsection\hspace{0.6em}} % code preceding the title

\usepackage{tocloft}[subfigure] % subfigure option only if using subfigure package
\renewcommand{\cfttoctitlefont} % ToC title
             {\usefont{T1}{opensans-TLF}{b}{n}\selectfont\huge}
% \renewcommand{\cftchapfont}% chapter titles
%              {\usefont{T1}{opensans-TLF}{b}{n}\selectfont}
\renewcommand{\cftsecfont} % section titles
             {\usefont{T1}{opensans-TLF}{b}{n}\selectfont}
\renewcommand{\cftsubsecfont} % subsection titles
             {\usefont{T1}{futs}{b}{n}\selectfont}
% \renewcommand{\cftchappagefont} % chapter page numbers
%              {\usefont{T1}{futs}{b}{n}\selectfont}
\renewcommand{\cftsecpagefont} % section page numbers
             {\cftsecfont} 
\renewcommand{\cftsubsecpagefont} % subsection page numbers
             {\cftsubsecfont}
  
\usepackage[activate={true,nocompatibility},final,tracking=true,kerning=true,spacing=true]{microtype}
\SetProtrusion{encoding={*},family={futs},series={*},size={6,7}}
                {1={ ,750},2={ ,500},3={ ,500},4={ ,500},5={ ,500},
                6={ ,500},7={ ,600},8={ ,500},9={ ,500},0={ ,500}}
\SetExtraKerning[unit=space]
    {encoding={*}, family={futs}, series={*}, size={footnotesize,small,normalsize}}
    {\textendash={400,400}, % en-dash, add more space around it
     "28={ ,150}, % left bracket, add space from right
     "29={150, }, % right bracket, add space from left
     \textquotedblleft={ ,150}, % left quotation mark, space from right
     \textquotedblright={150, }} % right quotation mark, space from left
\SetExtraKerning[unit=space]
   {encoding={*}, family={opensans-TLF}, series={b}, size={large,Large}}
   {1={-200,-200}, 
    \textendash={400,400}}
\SetTracking{encoding={*}, shape=sc}{25}

% ========================
%      MATH-Y STUFF
% ========================

\usepackage{mathtools} % loads asmmath
\usepackage{amssymb} % more symbols.. 
\newcommand{\Mod}[1]{\ (\mathrm{mod}\ #1)} % better modulo
\newcommand\Item[1][]{% equation alignment on enumeration
  \ifx\relax#1\relax  \item \else \item[#1] \fi
  \abovedisplayskip=0pt\abovedisplayshortskip=0pt~\vspace*{-\baselineskip}}

% OPTIONAL 
% \usepackage{nicefrac,xfrac} % prettify small fractions
% \usepackage{esint} % more integral signs
% \usepackage{ntheorem} % enhanced theorem env
% \usepackage{numprint} % print big numbers....

% pseudocode and code input
% \usepackage[noend]{algpseudocode}
% \usepackage{minted}
% \renewcommand\algorithmicthen{}
% \renewcommand\algorithmicdo{}
% \MakeRobust{\Call}

% physics package but better:
% \DeclarePairedDelimiter{\bra}{\langle}{\rvert}%
% \DeclarePairedDelimiter{\ket}{\lvert}{\rangle}%
% \DeclarePairedDelimiterX\innerp[2]{\langle}{\rangle}{#1\delimsize\vert\mathopen{}#2}%
% \DeclarePairedDelimiterX\braket[2]{\langle}{\rangle}{#1\delimsize\vert\mathopen{}#2}%
% \DeclarePairedDelimiterX\braketOP[3]{\langle}{\rangle}{#1\,\delimsize\vert\,\mathopen{}#2\,\delimsize\vert\,\mathopen{}#3}%
% \DeclarePairedDelimiterX\ketbra[2]{\lvert}{\rvert}{#1\delimsize\rangle\!\delimsize\langle#2}%
% \DeclarePairedDelimiterX\outerp[2]{\lvert}{\rvert}{#1\delimsize\rangle\!\delimsize\langle#2}%
% \DeclarePairedDelimiterX\projector[1]{\lvert}{\rvert}{#1\delimsize\rangle\!\delimsize\langle#1}%

%\graphicspath{ { figs/ } }     

\begin{document}

\maketitle

\pagenumbering{roman}
\tableofcontents

\newpage
\pagenumbering{arabic}

\section{Introduction}
\label{sec:introduction}

\subsection{Course Overview}
\label{subsec:overview}

This course is designed to:

\begin{enumerate}
\item familiarize you with \emph{what} models of decision making exist in
  economics, and
\item develop skills in \emph{how} to model decision making.
\end{enumerate}

This will be done through lectures, problem sets, and presenting papers on
topics in decision theory. Students will also research new models or develop
economic questions and answer them.

\subsection{Course Outline}
\label{sec:outline}

\begin{enumerate}
\item Introductions to decision science
  \begin{enumerate}
  \item What is ``decision theory''?
  \item Should we study what people do or study what they \emph{should} do?
  \end{enumerate}
  
\item Standard choice
  \begin{enumerate}
  \item Model primitives: preferences or choices?
  \item What does ``rationality'' look like?
  \end{enumerate}
  
\item Choice under uncertainty
  \begin{enumerate}
  \item How do/should we make choices when the state of the world is unknown?
  \item Philosophical approaches to uncertainty (ignorance)
  \item Subjective expected utility and ambiguity aversion
  \item Case-based decision theory
  \end{enumerate}
  
\item Choice under risk
  \begin{enumerate}
  \item How do/should we make choices when there is risk, but we know objective
    probabilities?
  \item Choice axioms: normative or descriptive?
  \item Rank dependent utility, cumulative prospect theory, salience
  \end{enumerate}

\item Limited Attention
  \begin{enumerate}
  \item Humans have limited information processing capacity
  \item How does this affect choice?
  \item What does ``rationality'' look like if we allow for people to not pay
    full attention?
  \item Sparse maximization, salience, limited consideration, rational inattention
  \end{enumerate}

\item Heuristics
  \begin{enumerate}
  \item In (nearly) all of the above, we typically assume a certain level of
    ``rationality''
  \item What happens if people use decision shortcuts, or heuristics?
  \end{enumerate}
\end{enumerate}


\subsection{On Models}
\label{subsec:models}

Decision theorists and economists are interested in both how people make
decisions and how they \emph{should} make decisions (if they were ideally
rational). This is typically done by producing models. But what \emph{is}
a model?

A model is a theoretical representation of reality; they come in many
forms, often mathematical. Decision-making models are the primary interest
of decision theorists. They often satisfy the following properties:
\begin{itemize}
\item \textbf{Simplified:} certain aspects that exist in real life are
  excluded from the model.
\item \textbf{Tractable:} it can be solved using relatively simple methods.
\item \textbf{As-if:} decision makers can be modeled ``as if'' they behave
  according to the model.
\end{itemize}

A good method for building a model might begin by constructing a highly
simplified example. This can help to develop the foundational understanding
necessary to work out a toy model.

Once you have a good starting point, work to generalize the model. Consider the
simplifications you made to produce the example, and start to revert them.  The
goal of generalizing the model is to increase its flexibility; the level of
``flexibility'' may depend on the particular problem of interest.

Once understood, consult the literature on the problem. How have other people
chosen to construct their models? How are they similar and different from yours,
and why?

\section{Standard choice}
\label{sec:standard-choice}


\subsection{Resnick (1987), ch. 1}
\label{subsec:resnick}

It's important to note that Resnick approaches decision theory from his
background as a philosopher rather than an economist. This perspective
influences both his terminology and objectives. In the preface, he
contextualizes decision theory as a broad field that encompasses utility theory,
game theory, and social choice theory. His stated aim is to ``put forward an
exposition of the theory that pays particular attention to matters of logical
and philosophical interest,'' reflecting his focus on theoretical foundations and
their associated philosophical impact rather than economic applications.

Chapter one then opens by defining decision theory. It is usual to divide
decision theory into two main branches: descriptive decision theory, which seeks
to find out how decisions \emph{are} made; and normative (or prescriptive)
decision theory, which seeks to prescribe how decisions \emph{ought} to be
made. The latter study ideally rational agents. That is not to say these
branches are completely separated, however\textemdash for instance, when
studying chess players' strategies to develop guidelines for novice players. 

\subsection{Model primitives}
\label{subsec:primitives}

Preferences vs. Choices.

First, properties of preferences.

Utility? not always the primitive. Often, it is a preference relation.  But
under certain assumptions, you can represent a preference using a utility
function.

\subsection{Preferences}
\label{subsec:preferences}

\subsubsection{Utility representations}
\label{subsubsec:utility-representations}

\subsubsection{From Mas-Colell, Whinston, and Green}
\label{subsubsec:preference-utility}

\subsection{Choices}
\label{subsec:choices}

\subsection{WARP}
\label{subsec:warp}

\subsection{Preferences and WARP, connected}
\label{subsec:preferences-and-warp}

\subsection{Classical demand theory}
\label{subsec:classical-demand-theory}

\section{Decision making under uncertainty}
\label{sec:uncertainty}

Most decisions we face on a day-to-day basis involve some degree of
uncertainty. Whether that comes in deciding which stock to invest in, what movie
to watch, or what job to accept, the actual probability of each outcome is not
known. Consider: when we discuss risk, everyone agrees on the probabilities;
when discussing uncertainty, there is disagreement. That is, when analyzing
someone's actions, we cannot make any statements \emph{a priori} regarding their
beliefs.

\subsection{Back to Resnick}
\label{subsec:resnick-uncertainty}

\subsection{Subjective expected utility\textemdash Savage (1954)}
\label{subsec:savage-seu}

Karni (2005)

\subsubsection{SEU properties}
\label{subsubsec:seu-properties}

\subsubsection{SEU, formally}
\label{subsubsec:seu-proper}

\subsection{Ellsberg paradox}
\label{subsec:ellsberg-paradox}

\subsection{Maxmin expected utility\textemdash Gilboa and Schmeidler}
\label{subsec:gs-maxmin}

\subsection{Awareness of unawareness\textemdash Karni and Vier\o \,(2017)}
\label{subsec:kv-awareness}

\subsection{Case-based decision theory\textemdash Gilboa and Schmeidler (1995)}
\label{subsec:gs-cbdt}

\section{Decision making under risk}
\label{sec:risk}

\subsection{Expected utility}
\label{subsec:expected-utility}

\subsection{Preferences for risk}
\label{subsec:risk-pref}

\subsection{Kahneman and Tversky (1979)}
\label{subsec:kt}

\subsection{Rank dependent utility\textemdash Quiggin (1982, 1993)}
\label{subsec:quiggin-rdu}

\subsection{Cumulative prospect theory}
\label{subsec:cpt}

\subsection{CPT axioms\textemdash Chateauneuf and Wakker (1999)}
\label{subsec:cpt-axioms}

\subsection{Disappointment aversion\textemdash Gul (1991)}
\label{subsec:gul-dis-aversion}

\section{Limited attention}
\label{sec:limited-attention}

\subsection{Sparse maximization\textemdash Gabaix (2014)}
\label{subsec:label}


\end{document}

\documentclass[letterpaper, 12pt]{article}
\usepackage{hyperref}
\usepackage{graphicx}

\title{Content}
\author{Nikhil Panickssery}
\date{\today}

% ========================
%    FONTS AND ENCODING
% ========================

\usepackage{fourier}
\usepackage{opensans}
\usepackage[T1]{fontenc}

\usepackage[DIV=14,BCOR=2mm,headinclude=true,footinclude=false]{typearea}
\makeatletter
\if@twoside % commands below work only for twoside option of \documentclass
    \newlength{\textblockoffset}
    \setlength{\textblockoffset}{12mm}
    \addtolength{\hoffset}{\textblockoffset}
    \addtolength{\evensidemargin}{-2.0\textblockoffset}
\fi
\makeatother

\usepackage{titlesec} % www.ctan.org/tex-archive/macros/latex/contrib/titlesec/titlesec.pdf
\titleformat{name=\section}[hang]{
  \usefont{T1}{opensans-TLF}{b}{n}\selectfont}
  {} % label
  {0em} % horizontal separation between label and title body
  {\hspace{-0.4pt}\Large \thesection\hspace{0.6em}} % code preceding the title
\titleformat{name=\section,numberless}[hang]{
  \usefont{T1}{opensans-TLF}{b}{n}\selectfont}
  {} % label
  {0em} % horizontal separation between label and title body
  {\hspace{-0.4pt}\Large } % code preceding the title
\titleformat{\subsection}[hang]{
  \usefont{T1}{opensans-TLF}{b}{n}\selectfont}
  {} % label
  {0em} % horizontal separation between label and title body
  {\hspace{-0.4pt}\large \thesubsection\hspace{0.6em}} % code preceding the title
\titleformat{name=\subsection,numberless}[hang]{
  \usefont{T1}{opensans-TLF}{b}{n}\selectfont}
  {} % label
  {0em} % horizontal separation between label and title body
  {\hspace{-0.4pt}\large } % code preceding the title
\titleformat{\subsubsection}[hang]{
  \usefont{T1}{opensans-TLF}{b}{n}\selectfont}
  {} % label
  {0em} % horizontal separation between label and title body
  {\hspace{-0.4pt}\large \thesubsubsection\hspace{0.6em}} % code preceding the title

\usepackage{tocloft}[subfigure] % subfigure option only if using subfigure package
\renewcommand{\cfttoctitlefont} % ToC title
             {\usefont{T1}{opensans-TLF}{b}{n}\selectfont\huge}
% \renewcommand{\cftchapfont}% chapter titles
%              {\usefont{T1}{opensans-TLF}{b}{n}\selectfont}
\renewcommand{\cftsecfont} % section titles
             {\usefont{T1}{opensans-TLF}{b}{n}\selectfont}
\renewcommand{\cftsubsecfont} % subsection titles
             {\usefont{T1}{futs}{b}{n}\selectfont}
% \renewcommand{\cftchappagefont} % chapter page numbers
%              {\usefont{T1}{futs}{b}{n}\selectfont}
\renewcommand{\cftsecpagefont} % section page numbers
             {\cftsecfont} 
\renewcommand{\cftsubsecpagefont} % subsection page numbers
             {\cftsubsecfont}
  
\usepackage[activate={true,nocompatibility},final,tracking=true,kerning=true,spacing=true]{microtype}
\SetProtrusion{encoding={*},family={futs},series={*},size={6,7}}
                {1={ ,750},2={ ,500},3={ ,500},4={ ,500},5={ ,500},
                6={ ,500},7={ ,600},8={ ,500},9={ ,500},0={ ,500}}
\SetExtraKerning[unit=space]
    {encoding={*}, family={futs}, series={*}, size={footnotesize,small,normalsize}}
    {\textendash={400,400}, % en-dash, add more space around it
     "28={ ,150}, % left bracket, add space from right
     "29={150, }, % right bracket, add space from left
     \textquotedblleft={ ,150}, % left quotation mark, space from right
     \textquotedblright={150, }} % right quotation mark, space from left
\SetExtraKerning[unit=space]
   {encoding={*}, family={opensans-TLF}, series={b}, size={large,Large}}
   {1={-200,-200}, 
    \textendash={400,400}}
\SetTracking{encoding={*}, shape=sc}{25}

% ========================
%      MATH-Y STUFF
% ========================

\usepackage{mathtools} % loads asmmath
\usepackage{amssymb} % more symbols.. 
\newcommand{\Mod}[1]{\ (\mathrm{mod}\ #1)} % better modulo
\newcommand\Item[1][]{% equation alignment on enumeration
  \ifx\relax#1\relax  \item \else \item[#1] \fi
  \abovedisplayskip=0pt\abovedisplayshortskip=0pt~\vspace*{-\baselineskip}}

% OPTIONAL 
% \usepackage{nicefrac,xfrac} % prettify small fractions
% \usepackage{esint} % more integral signs
% \usepackage{ntheorem} % enhanced theorem env
% \usepackage{numprint} % print big numbers....

% pseudocode and code input
% \usepackage[noend]{algpseudocode}
% \usepackage{minted}
% \renewcommand\algorithmicthen{}
% \renewcommand\algorithmicdo{}
% \MakeRobust{\Call}

% physics package but better:
% \DeclarePairedDelimiter{\bra}{\langle}{\rvert}%
% \DeclarePairedDelimiter{\ket}{\lvert}{\rangle}%
% \DeclarePairedDelimiterX\innerp[2]{\langle}{\rangle}{#1\delimsize\vert\mathopen{}#2}%
% \DeclarePairedDelimiterX\braket[2]{\langle}{\rangle}{#1\delimsize\vert\mathopen{}#2}%
% \DeclarePairedDelimiterX\braketOP[3]{\langle}{\rangle}{#1\,\delimsize\vert\,\mathopen{}#2\,\delimsize\vert\,\mathopen{}#3}%
% \DeclarePairedDelimiterX\ketbra[2]{\lvert}{\rvert}{#1\delimsize\rangle\!\delimsize\langle#2}%
% \DeclarePairedDelimiterX\outerp[2]{\lvert}{\rvert}{#1\delimsize\rangle\!\delimsize\langle#2}%
% \DeclarePairedDelimiterX\projector[1]{\lvert}{\rvert}{#1\delimsize\rangle\!\delimsize\langle#1}%

%\graphicspath{ { figs/ } }     

\begin{document}

\maketitle

\section{Introduction}
\label{sec:introduction}

This course is designed to:

\begin{enumerate}
\item familiarize you with \emph{what} models of decision making exist in
  economics, and
\item develop skills in \emph{how} to model decision making.
\end{enumerate}

\section{Models}
\label{sec:models}

The following is the list of models we will analyze for the course:

\begin{enumerate}
\item k
\end{enumerate}

\section{Uncertainty}
\label{sec:uncertainty}

Most decisions we face on a day-to-day basis involve some degree of
uncertainty. Whether that comes in which stock to invest in, what movie to
watch, or what job to accept, the actual probability of each outcome is not
known. Consider: when we discuss risk, everyone agrees on the probabilities;
when discussing uncertainty, there is disagreement. That is, when analyzing
someone's actions, we cannot make any statements \emph{a priori} regarding their
beliefs.

\subsection{Resnick}
\label{subsec:resnick}

\subsection{Savage and SEU}
\label{subsec:savage}

\subsection{Ellsberg}
\label{subsec:ellsberg}

SEU cannot account for ambiguity aversion... so we will try to model it. One way
is maxmin expected utility of Gilboa and Schmeidler (1989), where people do not
have a single coherent belief but a \emph{set} of beliefs. E.g., in the Ellsberg
example, you can think that the probability that the ball is Black is in a range
from . Assume the worst (pessimistic).

Formally:
\[V(f) = \min_{p \in C} [E_p(u(f))]\]

...horse-roulette.

\(f = (s_1, l_1; s_2, l_2)\) where $l_i$ is a lottery over monetary prizes.

Properties:
\begin{enumerate}
\item Rational (complete and transitive)
\item Independence:
  \[\forall f,g,h \text{ and } \alpha \in (0,1), f \succeq g \iff\]
  \[\alpha f + (1-\alpha)h \succeq \alpha g + (1-\alpha)h\]

  This should read: if act $f$ is preferred to act $g$, then any mixture of $f$ with a
  third act $h$ should be preferred to the same mixture of $g$ with $h$.
  
\item Non-Degeneracy, Continuity, Monotonicity
\end{enumerate}

\subsection{Karni}
\label{subsec:karni}

Awareness of unawareness...

Acts, mappings from known states to known consequences: $A$

Consequences we know are possible and can be clearly described: $C$

Unknown consequence, catch-all for ``none-of-the-above'': $x$

Extended consequences: $\hat{C} = C \cup \{x\}$

Augmented conceivable states: $\hat{C}^A \coloneq \{s : A \rightarrow \hat{C}\}$

Fully describable conceivable states: $C^A \coloneq \{s : A \rightarrow C\}$

Set of conceivable acts: $F \coloneq \{f : \hat{C}^A \rightarrow C\}$

Set of extended conceivable acts:
\[F^* \coloneq \{f^* : \hat{C}^A \rightarrow \hat{C} \mid
  f^{*-1}(x) \subseteq \hat{C}^A \setminus C^A \}\]

We consider lotteries over extended conceivable acts. Denote by $\Delta (F^*)$ the
set of all probability distributions on $F^*$, and by $\Delta (F)$ its subset of all
probability distributions on $F$. A generic element \(\mu \in \Delta (F^*)\) selects
an extended conceivable act in $F^*$ according to the distribution $\mu$.

Refer to elements of \(\Delta (F^*)\) by the name \emph{mixed extended conceivable acts}.
The set \(\Delta (F^*)\) of all such lotteries is the choice set.


\end{document}

% ========================
%      MATH-Y STUFF
% ========================

\usepackage{mathtools} % loads asmmath
\usepackage{amssymb} % more symbols.. 
\newcommand{\Mod}[1]{\ (\mathrm{mod}\ #1)} % better modulo
\newcommand\Item[1][]{% equation alignment on enumeration
  \ifx\relax#1\relax  \item \else \item[#1] \fi
  \abovedisplayskip=0pt\abovedisplayshortskip=0pt~\vspace*{-\baselineskip}}

% OPTIONAL 
% \usepackage{nicefrac,xfrac} % prettify small fractions
% \usepackage{esint} % more integral signs
% \usepackage{ntheorem} % enhanced theorem env
% \usepackage{numprint} % print big numbers....

% pseudocode and code input
% \usepackage[noend]{algpseudocode}
% \usepackage{minted}
% \renewcommand\algorithmicthen{}
% \renewcommand\algorithmicdo{}
% \MakeRobust{\Call}

% physics package but better:
% \DeclarePairedDelimiter{\bra}{\langle}{\rvert}%
% \DeclarePairedDelimiter{\ket}{\lvert}{\rangle}%
% \DeclarePairedDelimiterX\innerp[2]{\langle}{\rangle}{#1\delimsize\vert\mathopen{}#2}%
% \DeclarePairedDelimiterX\braket[2]{\langle}{\rangle}{#1\delimsize\vert\mathopen{}#2}%
% \DeclarePairedDelimiterX\braketOP[3]{\langle}{\rangle}{#1\,\delimsize\vert\,\mathopen{}#2\,\delimsize\vert\,\mathopen{}#3}%
% \DeclarePairedDelimiterX\ketbra[2]{\lvert}{\rvert}{#1\delimsize\rangle\!\delimsize\langle#2}%
% \DeclarePairedDelimiterX\outerp[2]{\lvert}{\rvert}{#1\delimsize\rangle\!\delimsize\langle#2}%
% \DeclarePairedDelimiterX\projector[1]{\lvert}{\rvert}{#1\delimsize\rangle\!\delimsize\langle#1}%